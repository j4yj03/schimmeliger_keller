% !TEX root = ../Thesis.tex
%%
%%  Hochschule für Technik und Wirtschaft Berlin --  Abschlussarbeit
%%
%%  Abstract - Englisch
%%
%%%%%%%%%%%%%%%%%%%%%%%%%%%%%%%%%%%%%%%%%%%%%%%%%%%%


\section*{Abstract}

This Master Thesis describes the implementation of a solution to control and monitor electric consumers via the Internet. Needs of this solution are defined by use of the Kano model. Existing technologies and standards are benchmarked by means of a cost-utility analysis. The solution that scores the highest value of benefit will be implemented typically.

A Zigbit Module forms the basis of the hardware. It bundles an AVR microcontroller and an IEEE 802.15.4 transceiver. Together with a self-dimensioned capacitive power supply it is mounted in a socket housing.

To be future-proof, the IPv6 protocol is used. The 6LoWPAN protocol handles the adaptation layer. Contiki is used as operating system. It is delivered with a ready-to-use web server application which is customized for the own purposes. The IEC 60870-5-104 protocol is implemented from scratch. It is based on TCP/IP and is used in the field of energy management systems. It is particularly suitable for automated access.

Via a public address given by IPv6 tunnel broker SixXS the solution is accessible worldwide. The electric consumer can be switched on and off by the means of a web browser or an energy management system.

The needs according to the Kano model are almost completeley achieved. It is possible to implement a solution consisting of a web server and IEC 60870-5-104 application in ressource constraint environments. Possible applications for such a solution are in the field of home automation and smart grid.


%% eof
