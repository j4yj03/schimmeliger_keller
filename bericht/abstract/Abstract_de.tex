% !TEX root = ../Thesis.tex
%%
%%  Hochschule für Technik und Wirtschaft Berlin --  Abschlussarbeit
%%
%%  Abstract - Deutsch
%%
%%%%%%%%%%%%%%%%%%%%%%%%%%%%%%%%%%%%%%%%%%%%%%%%%%%%


\section*{Kurzfassung}

Diese Arbeit beschreibt die Erstellung einer internetfähigen Steuerung für elektrische Verbraucher. Anforderungen an die Steuerung werden nach dem Kano-Modell definiert. Über eine Nutzwertanalyse werden vorhandene Techniken und Standards bewertet. Exemplarisch wird die Lösung mit dem größten Nutzwert implementiert.

Ein Zigbit-Modul, bestehend aus einem AVR Mikrocontroller und einem IEEE 802.15.4 Funkchip, bildet die Basis für die Hardware. Zusammen mit einem selbst dimensioniertem Kondensatornetzteil wird das Modul in einem Steckdosengehäuse verbaut.

Um zukunftssicher zu sein, wird das Protokoll IPv6 eingesetzt. Die Adaptionsschicht übernimmt das Protokoll 6LoWPAN. Das verwendete Betriebssystem Contiki besitzt eine fertige Webserver-Applikation, die für die eigenen Zwecke angepasst wird. Das Protokoll IEC 60870-5-104 wird neu implementiert. Es basiert auf dem TCP/IP-Modell und wird vor allem im Umfeld von Energieleitsystemen eingesetzt. Es eignet sich besonders für einen automatisierten Zugriff.

Über eine öffentliche Adresse des IPv6-Tunnelbrokers SixXS ist die Steuerung weltweit erreichbar und der elektrische Verbraucher kann über einen Webbrowser oder von einem Energieleitsystem ein- und ausgeschaltet werden.

Die Anforderungen nach dem Kano-Modell wurden nahezu vollständig erfüllt. Die Implementierung eines Webservers und einer IEC 60870-5-104 Applikation ist mit den gegebenen limitierten Ressourcen möglich. Anwendungsmöglichkeiten für die Steuerung liegen im Bereich eHome und Smart Grid.

%Motivation, Fragestellung, Methodik, Ergebnisse, Schlussfolgerungen
