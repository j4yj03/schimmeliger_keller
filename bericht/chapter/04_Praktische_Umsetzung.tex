% !TEX root = ../Abschlussbericht_Schimmeliger_Keller.tex
%%
%%  Hochschule für Technik und Wirtschaft Berlin --  Projektabschlussbericht
%%
%% Kapitel 4 - Praktische Umsetzung
%%
%%

\chapter{Praktische Umsetzung} \label{Praktische Umsetzung}
\section{Verwendete Hardware} \label{Hardware}
\subsection{LoPy4-Development-Board} \label{LoPy4}

Das LoPy4 ist ein von Pycom Ltd. hergestelltes Development-Board für die Anwendungen im IoT-Bereich. Es unterstützt viele Schnittstellen und Übertragungsprotokolle, die für IoT-Anwendung von großer Relevanz sind. Die Abbildung \ref{fig:lopy-blockschaltbild} stellt die Funktionalitäten des LoPy4-Development-Boards in Form eines Blockdiagrammes  dar. 

\begin{figure}[h]
 \centering
 \includegraphics[width=1\textwidth]{pictures/blockdiagram_lopy}
 \caption[LoPy4-Blockdiagramm]{LoPy4-Blockdiagramm}\cite{Lopy2022}
 \label{fig:systemkonzept}
\end{figure}

Im Kern befindet sich ein Espressif ESP32 mit Xtensa dual-core-32-bit LX6 Mikroprozessor. Als Speicher verwendet es 520KB + 4MB RAM und einen externen 8MB Flash-Speicher, wohin auch der Code dann geladen wird. 
Zur Ein-/Ausgabe und der Steuerung von Sensoren und Aktoren, bietet das Board vielseitige Peripherien wie GPIO, ADC, DAC, SPI, I2C, UART und PWM an. 

Darüberhinaus verfügt das Entwicklungsboard über WLAN (802.11b/g/n/e/i), Bluetooth/Bluetooth Low Energy (BLE) v4.2, sowie LoRa und Sigfox Protokolle.  
Es lässt sich leicht mit einem Breadboard benutzen, um damit einfache Schaltungen, ohne zu löten, testen zu können. 

Für die WLAN und Bluetooth Anwendung kann man entweder die interne, im Board eingebaute, oder eine externe Antenne verwenden. Bei der Nutzung von LoRa und Sigfox muss man jedoch darauf achten, dass man unbedingt eine externe Antenne verwendet, denn sonst könnte damit der LoRa-Chip beschädigt und unbrauchbar gemacht werden. Dafür gibt es zwei Anschlussmöglichkeiten; einmal für das 868 MHz und einmal für das 433 MHz Frequenzband. 
Dafür wird der SX1276 LoRa-Transceiver-Chip der Firma Semtech verwendet.

Für die Spannungsversorgung wird lediglich ein Mikro-USB-Kabel, welches an eine USB-Schnittstelle am Computer das Board mit 3.3-5.5V versorgt, benötigt. Neben der Spannungsversorgung kann damit auch das Board direkt programmiert werden. 

Für die Programmierung des Boards kann man auf unterschiedliche Entwicklungsumgebungen zurückgreifen. So werden die Entwicklungsumgebungen wie Atom und Visual Studio Code (VSC) von dem Hersteller mit einem sogenannten Pymakr-Plugin für einen reibungslosen Upload und das Flashen des Boards unterstützt. Wenn das Benutzen des Plugins nicht möglich ist, wie unter anderem es teilweise bei uns der Fall war, so kann man auf das von PyCom angebotene File-Transfer-Protocol (FTP) zugreifen. Dafür muss man zunächst einmal einen FTP-Server auf dem Board erzeugen, mithilfe dessen man dann auf das Filesystem des Boards Zugriff hat. Damit der Code im Board läuft, muss man lediglich die jeweiligen Dateien (hauptsächlich die main.py Datei) auf das Board mit FTP-Befehlen in den Flash-Speicher kopieren. 

Für die Unterstützung von LoRa bietet Pycom zwei Möglichkeiten an: LoRa-RAW und LoRaWAN. 
Mit LoRa-RAW kann eine direkte Punkt-zu-Punkt Verbindung für die Kommunikation zwischen zwei LoRa-Knoten hergestellt werden. Mit LoRaWAN kann darüberhinaus der LoRa-Knoten in das TheThingsNetwork (TTN) oder das Chirpstack-Netzwerk eingebunden werden. Die Bandbreite kann zwischen den Werten 125, 250 und 500 kHz variiert werden. Außerdem kann man den Spreizfaktor zwischen 7 und 12 auswählen. Für die Synchronisation kann die Preamble mit der Anzahl der Chirps angepasst werden, wobei der Standardwert bei acht liegt. Wieviele „Chips” ein Symbol definieren, kann man mit der Coderate einstellen.
Darüberhinaus ist zu erwähnen, dass PyCom mit PyMesh, auch die Möglichkeit eines LoRa-Mesh-Netzwerkes auf der MAC-Ebene anbietet. Es ist dabei wichtig, dass alle LoRa-Knoten das gleiche Frequenzband, Spreizfaktor und Bandbreite verwenden. Es verwendet dazu ein hierarchisches Mesh-Netzwerkmodell.

\subsection{Mögliche Schaltung zum Selbstbauen} \label{LoPy4}

\begin{center}
	\begin{figure}[h]
	 
	 \noindent\makebox[\textwidth]{\includegraphics[width=1.1\textwidth]{pictures/sensor_schematic}}
	 \caption[Möglicher Schaltplan für einen Selbstbau]{Möglicher Schaltplan für einen Selbstbau}
	 \label{fig:dhtkommunikation}
	\end{figure}
\end{center}

Der Microcontroller kann ein beliebiger ESP32 basierender Microcontroller sein, wobei der LoPy4 das LoRa Modul bereits integriert hat. Der Transistor kann ein generischern NPN-Transistor sein und der Wiederstand R2 ist theoretisch auch optional, wenn der Microcontroller einen internen Pull-Up Wiederstand dazuschalten kann.

\subsection{DHT Sensormodul} \label{DHT}

Bei dem verwendeten Sensor handelt es sich um einen sog. DHT Sensor. Diesen gibt es in zwei verschiedenen Ausführungen, DHT11 und DHT22, wobei sich diese im auswertbaren Messbereich, der Messgenauigkeit und im Preis unterscheiden.

\begin{adjustwidth}{-1in}{-1in}% adjust the L and R margins by -1 inch
	\begin{center}
	
	        \begin{tabular}{ccc}
			\toprule
			 & \textbf{DHT11} & \textbf{DHT22}\\

			\midrule
			Betriebsspannung & \multicolumn{2}{c}{3 \dots 5 V DC}\\
			Stromverbrauch & \multicolumn{2}{c}{max. 2,5 mA während der Konvertierung}\\
			Temperaturbereich & -20\dots 60 °C & -40\dots 80 °C  \\
			Temperatur Genauigkeit & ± 2,0 °C & ± 0,5 °C\\
			Feuchtigkeit Messbereich & 20\%\dots90\% RH & 0\%\dots99,9\% RH\\
			Feuchtigkeit Genauigkeit & ± 5,0\% RH & ± 2\dots5\%\\
			Abtastrate & 1 Hz & 0,5 Hz \\

			\midrule
			Preis (bei reichelt elektronik) & 1,80 € & 6,80 €\\

			\bottomrule
	
	        \end{tabular}
		\label{}
		\captionof{table}{Vergleich DHT11 zu DHT22} \label{tab:vergleichDHT} 
	\end{center}
\end{adjustwidth}

Anzumerken ist noch, dass der DHT22 Sensorungefähr das doppelte Volumen des DHT11 Sensors hat. Interessant ist auch, dass der DHT11 Sensor ungefähr doppelt so häufig angesprochen werden kann, was vermutlich auf seiner weniger komplexen Schaltung beruht.\\
Für die meisten Anwendungsfälle würde wahrscheinlich ein DHT11 Sensor (oder mehrere parallel geschaltete DHT11 Sensoren) ausreichen, um ein weitgehend zuverlässiges Ergebnis zu erhalten.

\subsubsection{Kommunikation mit dem DHT Sensor\cite{dht}} 

Die Kommunikation mit dem DHT Sensor erfolgt über eine einzelne Leitung, was zum einen die Kosten reduziert, zum anderen aber auch die Reichweite der störungfreien Übertragung erhöht. Um den Sensor anzusprechen und die Daten anschließend auszulesen, muss der Datenstrom kodiert werden. Dies geschieht über ein Protokoll, welches die Kommunikation in drei Schritte aufteilt:
\begin{itemize} 
	\item Request (die Anfrage)
	\item Repsonse (die Antwort des Sensors)
	\item Data (die vom Sensor übertragenen Daten)
\end{itemize}


\begin{center}
	\begin{figure}[h]
	 
	 \noindent\makebox[\textwidth]{\includegraphics[width=1.1\textwidth]{pictures/dht_kommunikation1}}
	 \caption[DHT Kommunikation]{DHT Kommunikation}
	 \label{fig:dhtkommunikation}
	\end{figure}
\end{center}


Um den DHT Sensor dazu zu bringen, Messewerte zu senden, zieht der Microcontroller den Datenbus für mindestens 18ms auf Masse, um ihn anschließend für 40µs wieder auf Versorgungsspannungsniveau zu ziehen. Dadurch versteht der DHT Sensor, dass er beginnen soll Messwerte zu sammeln.\\
Der DHT Sensor antwortet zunächst mit einer Sequenz, bestehend aus einem ca. 54µs langen LOW und anschließend 80µs HIGH. 

\newpage

Nachfolgend sendet der DHT Sensor fünf Datenpakete, bestehend aus jeweils 8 Bit, welche die einzelnen Sensor Messwerte mit einer Prüfsumme darstellen. Insgesammt sendet der DHT Sensor somit 40 Bit.

\begin{center}
	\begin{figure}[h]
	 
	 \noindent\makebox[\textwidth]{\includegraphics[width=0.9\textwidth]{pictures/dht11_dataframe}}
	 \caption[DHT Paketstruktur]{DHT Paketstruktur}
	 \label{fig:dhtpaketstruktur}
	\end{figure}
\end{center}

Die einzelnen Sensor Messwerte sind noch in Integral Anteil und Dezimal Anteil unterteilt, wobei die einzelnen Bits sich durch die dauer des HIGH Signals nach einem 54µs langem LOW Singal unterscheiden.

\begin{center}
	\begin{figure}[h]
	 
	 \noindent\makebox[\textwidth]{\includegraphics[width=0.9\textwidth]{pictures/dht_kommunikation2}}
	 \caption[DHT Bit Identifikation]{DHT Bit Identifikation}
	 \label{fig:dhtbits}
	\end{figure}
\end{center}

Abschließend sendet der DHT Sensor ein ca. 54µs langes LOW Signal, wonach der Bus wieder auf Versorgungsspannungsniveau gezogen wird und der Sensor in den Idle Mode geht.



\newpage

\subsection{Restliche Hardware} \label{Restliche Hardware}


\begin{itemize} 
	\item \textbf{Antenne:}  Bei der verwendeten Antenne handelt es sich um eine Multiband Antenne, welche für mehrere Freuquenzbänder (unter anderem das LoRa Freuquenzband) genutz werden kann.
	\item \textbf{USB Kabel:} Wir verwenden ein Daten-USB Kabel welches besonders geschirmt ist und eine maximal Länge von ca. 10cm aufweist. Wir hatten teilweise Probleme mit anderen USB Kabeln.
\end{itemize}



\section{Beschreibung der Software} \label{Software}

Für unser Projekt haben wir drei verschiedene, miteinander interagierende Software Komponenten realisiert, welche über eine Schnittstelle (Interface) miteinander kommunizieren. 
Der Vorteil einer solchen Architektur ist, dass die einzelnen Komponenten sich unter umständen wiederverwenden lassen und sich im Idealfall so eine Software modular aufbauen lässt.
Da wir als Programmiersprache ausschließlich Python bzw. Micropython verwendet haben, könnte man argumentieren, dass unsere Software automatisch Modular ist, da sich in der Theorie alle programmierten Komponenten in Python wiederverwenden lassen.
Dies ist aber sehr verallgemeinert gesprochen, da gerade die Programmierung der Mikrocontroller definitiv auch Individualsoftware benötigt, welche sich aber immerhin nicht nur auf einer einzelnen Mikrocontrollerfamilie ausführen funktionieren würde.
Anzumerken ist noch, dass für unsere finale Version des Projektes vermutlich nur eine einzelne Softwarekomponente notwendig wäre.\\
Für die Programmierung der Mikrocontroller verwenden wir die Programmiersprache Micropython, welche eine schlanke und schnelle Implementation der Programmiersprache Python ist, welche für Mikrocontroller optimiert wurde.

\newpage


\subsection{1. Komponente: Sensoransteuerung und der Versand der Daten mittels LoRa(WAN)} \label{Sender}


\begin{center}
	\begin{figure}[h]
	 
	 \noindent\makebox[\textwidth]{\includegraphics[width=0.35\textwidth]{pictures/sens_read_lora_send}}
	 \caption[PAP komponente 1]{Programm Ablauf: Komponente 1}
	 \label{fig:lorasendsensorread}
	\end{figure}
\end{center}

Zunächst wird der Microcontroller initialisiert, wobei er standartmäßig mit allen Modulen (WiFi, LoRa, etc.) im aktivierten Zustand startet. Dieser Umstand beruht darauf, dass wir die PyCom Plattform nutzen, welche wenig vorab Konfiguration ermöglicht.\\
Jegliche Konfigurationsmöglichkeiten unserer Programme haben wir in extern Dateien im JSON Format abgelegt, welche im nächsten Schritt eingelesen werden. In der Konfigurationsdatei können unter Anderem die Dauer, die der Microcontroller im Schlafmodus verbringen soll, sowie die einzelnen Sensoren definiert werden.\\
Um das anbinden mehrere Sensonren zu ermöglichen haben wir uns dazu entschieden, eine Klasse zu programmieren, welche eben dieses Sensorobjekt darstellen soll. Der Treiber wurde von einem Benutzer auf github veröffentlicht\cite{dhtlib}, funktioniert aber im Prinzip nach dem oben genannten Verfahren, was bedeutet, dass zunächst eine gewisse Zeit gewartet wird, anschließend das Bus auf GND Niveau gezogen wird, um den Sensor mitzuteilen, dass er doch bitte anfängt Messwerte zu sammeln. Nachdem der Sensor eine Antwort gegeben hat, werden die empfangenen Bits zunächst alle gesammelt und anschließend in 5 einzelne Bytepakete decodiert. Abschließend prüft der Treiber mithilfer der Prüfsumme, ob die empfangenen Daten Sinn machen.\\ Je nach Sensortyp (DHT11/22) werden die empfangenen Daten nochmals in den richtigen Wertebereich \grqq verschoben\grqq.\\
Anschließend werden für alle initialisierten Sensoren, über den Treiber die aktuellen Messwerte eingelesen und in ein bytearray verpackt, wobei hier schon das erste mal geschaut wird, ob die gemessen Sensorwerte überhaupt Sinn ergeben bzw. in dem vom Hersteller angebenen Bereich fallen, da es selbst bei der korrekten Dekodierung immernoch zu unsinnigen Werten kommen kann. 
Wurde dieser Test erfolgreich bestanden wird im Microcontroller über das LoRa Modul über eine Bibliothek initialisiert und die ermittelten Werte werden versendet.\\ Ein weiterer Bestandteil des bytearrays ist die aktuelle Systemzeit, welche verwendet wird, um im späteren Verlauf ermitteln zu können, ob es sich wirklich um ein neues empfangenes Datenpaket handelt, oder ob das Paket in irgendeinem Puffer zunächst verloren gegangen ist und zufällig wieder ins Tageslicht gerückt ist.\\
Zu guter Letzt wird der Microcontroller in den Schlafmodus versetzt, wobei die Zeit, die er im stromsparenden Modus verbringt benutzerdefiniert ist. Anzumerken ist hier noch, dass der Sensor über einen Transistor ausgeschaltet wird um noch etwas mehr Strom zu sparen und bei der Neuinitialisierung wieder angeschaltet wird. Die Systemzeit läuft auch im Schlafmodus weiter, solange der Microcontroller mit Strom versorgt wird, nur das LoRa-Modul muss neu initialisiert werden.\\

\newpage

\subsection{2. Komponente: Emfangen der Daten und Versand ins Internet} \label{Empfänger}

\begin{center}
	\begin{figure}[h]
	 
	 \noindent\makebox[\textwidth]{\includegraphics[width=0.3\textwidth]{pictures/LoRaRead}}
	 \caption[PAP komponente 2]{Programm Ablauf: Komponente 2}
	 \label{fig:lorareadwifisend}
	\end{figure}
\end{center}

Diese Softwarekomponente läuft schlussendlich auf dem LoRa Gateway, bzw. in unserem Fall dem zweiten Microcontroller, welcher in unserem Fall den Empfänger bei der P2P Verbindung darstellt.
Wie bei der ersten Komponente, wird zunächst der Microcontroller initialisiert. Des Weiteren werden Konfigurationsdaten eingelesen, welche für die optionale WiFi Verbindung und das senden an die MQTT Broker benötigt werden. Möchte man die Sensordaten einfach nur über eine serielle Schnittstelle auslesen, können diese Felder entsprechend leer gelassen werden.
Außerdem wird in dieser Komponente das LoRa Modul direkt initialisiert, da dies nur einmalig geschehen muss.\\
Es beginnt eine Endlosschleife, welche in bestimmten Abständen schaut, ob an dem Socket neue, per LoRa emfpangene, Daten aufgetaucht sind. Sollte dies der fall sein, werden die empfangenens bytes decodiert und auf ihre plausibilität gebprüft. Machen die empfangenen Daten Sinn, werden sie ins JSON Format gebracht und über die Serielle Schnitstelle ausgegeben, sollte der Nutzer der Microcontroller mit dem Internet verbunden haben, werden die Daten per MQTT an die entsprechenden Broker gesendet.\\
In der Konfigurationsdatei kann der Nutzer bestimmte Schwellwerte definieren, bei denen die Daten an unterschiedliche Broker gesendet werden. Somit können auf dem einen Feed zunächst alle Daten gesammelt werden, auf einem anderen wiederum, nur Daten die einen wirklich interessieren.

\subsection{3. Komponente: Manuelles abrufen und versenden der veröffentlichten Daten} \label{PubSub}

Möchte der Nutzer seine Daten zunächst nur lokal verwalten, hat er dennoch die Möglichkeit mithilfe der beiden folgende Skripte die Daten Nachträglich per MQTT an einen Broker zu schicken, bzw. den entsprechenden MQTT Broker zu subscriben, um seine Daten an einer anderen Stelle für die weitere Verabeitung zu sammeln.\\
Dafür haben wir eine Pythonklasse bereitgestellt, welche es dem Nutzer ermöglicht, entweder ein MQTT Publisher und/oder Subscriber Objekt zu erzeugen.\\
Im Falle des Publishers hat man die Möglichkeit, mehrere Schwellwerte in einer Konfigurationsdatei abzulegen, sodass je nach Bedarf ein oder mehrere Feeds mit Daten gefüttert werden können.\\\\
Im Prinzip laufen beide Programme so, dass innerhalb einer Endlosschleife, entweder an der Seriellen Schnittstelle oder am MQTT Broker angefragt wird, ob neue Datensätze zur verfügung stehen. Beim Publisher gibt es noch die besonderheit, dass dieser die Daten erneut auf ihre Plausibilität prüft, wofür unter anderen ein Zeitstempel hilfreich ist, um zu erkennen, dass es sich um einen neuen Datensatz handelt.\\
Ist dies der Fall, werden die Daten an den Broker gesendet.\\
Der Subscriber gibt neue Datensätze in der Konsole/Serielle Schnittstelle aus, sodass diese ggf. weiter verarbeitet werden können.

\begin{center}
	\begin{figure}[h]
	 
	 \noindent\makebox[\textwidth]{\includegraphics[width=0.9\textwidth]{pictures/MQTTpubsub}}
	 \caption[PAP komponente 3]{Programm Ablauf: Komponente 3}
	 \label{fig:MQTTpubsub}
	\end{figure}
\end{center}


\section{Visualisierung der Sensordaten} \label{Dashboard und Visualisierung}

Für die Visualisierung unserer Sensordaten haben wir auf \textit{adafruit.io} einige Feeds erstellt welche zum Einen die Rohdaten, also alle Sensordaten, zum Anderen nur mit Daten, welche einen bestimmten Schwellwert überschritten haben, gefüttert werden. Die Feeds sind frei zugänglich und können von jedem gefüttert und betrachtet werden.\\
Zur Visualisierung haben wir ein Dashboard erstellt, welches alle Feed in einem Liniendiagramm darstellt.

\newpage


\section{Berechnung der Laufzeit im Batteriebetrieb} \label{Simulation}

\begin{center}
	\begin{figure}[h]
	 
	 \noindent\makebox[\textwidth]{\includegraphics[width=1.1\textwidth]{pictures/programmzyklus}}
	 \caption[Stromverbrauch während eines Programmzyklus]{Stromverbrauch während eines Programmzyklus}
	 \label{fig:stromzyklus}
	\end{figure}
\end{center}

Der durchschnittliche Stromverbrauch je Programmzyklus ergibt sich somit aus dem Tastverhältnis zwischen aktivem und passivem Modus (duty cycle).

\[I_{avg} = \frac{I_{aktiv}\cdot t_{aktiv} + I_{passiv} \cdot t_{passiv}}{t_{aktiv} + t_{passiv}}\]

Anzumerken ist hier jedoch noch, dass der Microcontroller eine gewisse Zeit und einen gewissen Strom benötigt, um den Modus zu wechseln. Diese Kennzahlen sollten für ein präzises Modell aus dem Datenblatt entnommen werden, was wir für unser Projekt erstmal vernachlässigt haben.

\subsubsection{Berechnung der durchschnittlichen Laufzeit im Bettriebetrieb}

Die Laufzeit im Betteriebetrieb lässt sich über die Kapazität der Batterie(n) und des durchschnittlichen Stromverbrauchs ermitteln

\[t_{Laufzeit} = \frac{C_{Batterie}}{I_{avg}}\]

Nachfolgend wird illustriert, wie sich die Batterielaufzeit mit der zunahme der Programmzyklusdauer erhöht. Die farbliche Markierung gibt an, ob der Sensor die gesammt Zeit angeschalten bleibt, oder über einen Schalter bzw. Transistor nur im Moment der Ermittlung der Messwerte eingeschaltet wird.

\begin{center}
	\begin{figure}[h]
	 
	 \noindent\makebox[\textwidth]{\includegraphics[width=1.1\textwidth]{pictures/batterielaufzeit}}
	 \caption[Potentiell mögliche Batterielaufzeiten in Abhängigkeit zur Programmzykluszeit]{Potentiell mögliche Batterielaufzeiten in Abhängigkeit zur Programmzykluszeit}
	 \label{fig:batterielaufzeit}
	\end{figure}
\end{center}

Zu sehen ist, dass die Batterielaufzeit in beiden Fällen mit der Zunahme der Programmzyklusdauert ansteigt, ab einer bestimmten Dauer, aber nach und nach weniger zunimmt.\\\\
Eine ausführlichere Simulation findet man im beiliegenden Jupyter Notebook.\cite{schimmel}

\newpage
