% !TEX root = ../Abschlussbericht_Schimmeliger_Keller.tex
%%
%%  Hochschule für Technik und Wirtschaft Berlin --  Projektabschlussbericht
%%
%% Kapitel 4 - Praktische Umsetzung
%%
%%

\chapter{Praktische Umsetzung} \label{Praktische Umsetzung}
\section{Verwendete Hardware} \label{Hardware}
\subsection{LoPy4-Development-Board} \label{LoPy4}



\subsection{DHT Sensormodul} \label{DHT}

\begin{center}
	\begin{figure}[h]
	 
	 \noindent\makebox[\textwidth]{\includegraphics[width=0.7\textwidth]{pictures/dht11_dataframe}}
	 \caption[DHT Paketstruktur]{DHT Paketstruktur}
	 \label{fig:zeitplanung}
	\end{figure}
\end{center}



\subsection{Restliche Hardware} \label{Restliche Hardware}




\section{Beschreibung der Software} \label{Software}

Für unser Projekt haben wir drei verschiedene, miteinander interagierende Software Komponenten realisiert, welche über eine Schnittstelle (Interface) miteinander kommunizieren. 
Der Vorteil einer solchen Architektur ist, dass die einzelnen Komponenten sich unter umständen wiederverwenden lassen und sich im Idealfall so eine Software modular aufbauen lässt.
Da wir als Programmiersprache ausschließlich Python bzw. Micropython verwendet haben, könnte man argumentieren, dass unsere Software automatisch Modular ist, da sich in der Theorie alle programmierten Komponenten in Python wiederverwenden lassen.
Dies ist aber sehr verallgemeinert gesprochen, da gerade die Programmierung der Mikrocontroller definitiv auch Individualsoftware benötigt, welche sich aber immerhin nicht nur auf einer einzelnen Mikrocontrollerfamilie ausführen funktionieren würde.
Anzumerken ist noch, dass für unsere finale Version des Projektes vermutlich nur eine einzelne Softwarekomponente notwendig wäre.


\subsection{1. Komponente: Sensoransteuerung und der Versand der Daten mittels LoRa(WAN)} \label{Sender}

Für die Programmierung der Mikrocontroller verwenden wir die Programmiersprache Micropython, welche eine schlanke und schnelle Implementation der Programmiersprache Python ist, welche für Mikrocontroller optimiert wurde.

\begin{center}
	\begin{figure}[h]
	 
	 \noindent\makebox[\textwidth]{\includegraphics[width=0.4\textwidth]{pictures/sens_read_lora_send}}
	 \caption[PAP komponente 1]{Programm Ablauf: Komponente 1}
	 \label{fig:zeitplanung}
	\end{figure}
\end{center}

\subsection{2. Komponente: Emfangen der Daten und Versand ins Internet} \label{Empfänger}

\ldots

\subsection{3. Komponente: Manuelles abrufen und versenden der veröffentlichten Daten} \label{PubSub}

\ldots


\section{Visualisierung der Sensordaten} \label{Dashboard und Visualisierung}

\ldots


\section{Berechnung der Laufzeit im Batteriebetrieb} \label{Simulation}

\ldots
