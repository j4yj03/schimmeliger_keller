% !TEX root = ../Abschlussbericht_Schimmeliger_Keller.tex
%%
%%  Hochschule für Technik und Wirtschaft Berlin --  Projektabschlussbericht
%%
%% Kapitel 5 - Fazit
%%
%%

\chapter{Fazit} \label{Fazit}

\section{Projektauswertung} \label{Projektauswertung}

\subsubsection{Sidney}
Insgesammt ist es ein sehr interessantes Projekt gewesen, für mich war es besonder interessant etwas mehr darüber zu erfahren, dass es scheinbar doch eine Menge Initiativen und Projekte gibt, die ihren Fokus auf das Commoning bzw. auf das bereitstellen von freien Inhalten legt.\\
Für mich persönlich ging es in der gesammten Veranstaltung zu sehr um ein Produkt, als um eine Vision, aber das ist ja bekanntlich geschmackssache.\\

\subsubsection{Ilja}





\section{Probleme und Herausforderungen} \label{Probleme und Herausforderungen}

\subsubsection{Sidney}
Neben kleineren Problemen wie z.B. einem falschen USB-Kabel, war es wie so häufig eher das Problem den Fokus auf die wichtigen Dinge zu legen. Dabei hat der Projektmanagement-Anteil  aberdefinitiv geholfen.

\subsubsection{Ilja}





\section{Ausblick} \label{Ausblick}

\subsubsection{Erweiterung von LoRa-LoRa zu LoRaWAN}
In unserem Projekt haben wir bisher leider kein richtiges Sensornetzwerk aufgebaut, sondern zwei Microcontroller über eine P2P LoRa Verbindung Kommunizieren lassen. Dies hat zum einen den Grund, dass wir selber kein LoRa Gateway aufbauen wollten, zum anderen aber auch kein umliegendes Gateway in erreichbarer Nähe haben.

\subsubsection{Benutzung von mehreren Sensorknoten an einem LoRa-Gateway}
Wenn wir es hinbekommen sollten ein eigenes Gateway bereitzustellen, welches auch unsere Publisher-Software implementiert hat, würden wir gerne mehrere Sensorknoten mit diesem Gateway verbinden, um so möglicherweise Unterschiede in den Messdaten zwischen den Microcontrollern zu entdecken.

\subsubsection{Erweiterung der Software zur Einbindung von mehreren Sensorkomponenten}
Derzeit haben bieten wir nur die Möglichkeit einen oder mehrere DHT Sensoren einzubinden. Wir würden gerne eine universelle Schnittstelle bereitstellen, die es ermöglicht eine Vielzahl von Sensoren anzubinden und auszuwerten.

\subsubsection{Batteriebetrieb in Hardware umsetzen}
Den Batteriebetrieb haben wir bisher nur Simuliert. Da wir keine geeigneten Konstruktion bauen konnten, die einen Batteriehalter ordnungsgemäß halten könnte. Zu einem Serienreifen Produkt gehört somit auch ein Gehäuse mit Battriehalter.

\subsubsection{Alternative Dashboard Möoglichkeiten}
Wir würden gerne weitere Dashboardmöglichkeiten evalueiren, da die kostenlose Version der adafruit.io Dashboards doch recht limiert ist. Ggf. ist es auch Sinnvoll eine eigene Dashboard Plattfrom bereitzustellen, da wir weiterhin den Ansatz der freien Inhalte verfolgen wollen.

\section{Abschlusswort} \label{Abschlusswort}

Peace

