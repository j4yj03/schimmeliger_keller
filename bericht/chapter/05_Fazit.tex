% !TEX root = ../Abschlussbericht_Schimmeliger_Keller.tex
%%
%%  Hochschule für Technik und Wirtschaft Berlin --  Projektabschlussbericht
%%
%% Kapitel 5 - Fazit
%%
%%

\chapter{Fazit} \label{Fazit}

\section{Projektauswertung} \label{Projektauswertung}

\subsubsection{Sidney}
Insgesammt war es für mich ein sehr interessantes Projekt gewesen. Für mich war es besonder interessant, zu erfahren, dass scheinbar doch eine Menge Initiativen und Projekte existieren, die ihren Fokus auf das Commoning bzw. auf das bereitstellen von freien Inhalten legen.\\
Für mich persönlich ging es in der Veranstaltung allerdings zu sehr um die Idee, ein Produkt zu vermarkten, wo wiederum jeder seinen eigenen Schwerpunkt legt.\\

\subsubsection{Ilja}
Durch die Durchführung des Projektes, angefangen mit der Auswahl der Projektteilnehmer und der Generierung der Projektideen in den einzelnen Workshops,
die von Hanna so toll und gruppendynamisch geleitet wurden, bis zur Durchführung der Umfragen bei Freunden und Bekannten, der Planung und schließlich der Umsetzung der Projektidee, habe ich persönlich viel mitgenommen und dazugelernt. \\
Außerdem hat es sehr viel Spaß gemacht, gemeinsam an etwas zu arbeiten, bei dem es im Vordergrund nicht um den Profit oder die maximale Gewinnerbringung geht, was im Grunde auch das Prinzip des Commoning ist, sondern, wie Sidney schon erwähnt hat, um eine gewisse Philosophie und eine Vision einen Mehrwert für die Community/Gesellschaft zu generieren. Es geht um aktive Mitbeteiligung, damit man nicht nur ein Konsument, sondern selbst auch ein Produzent sein kann und sich als Teil der Gesellschaft fühlt und anerkannt wird.

\section{Probleme und Herausforderungen} \label{Probleme und Herausforderungen}

\subsubsection{Sidney}
Neben kleineren Problemen wie z.B. einem falschen USB-Kabel, war es wie so häufig eher das Problem, den Fokus auf die wichtigen Dinge zu legen, wobei aber der Projektmanagement-Anteil definitiv geholfen hat, etwas struktur in den ganzen Djungel zu bringen.

\subsubsection{Ilja}
Die ersten Herausforderungen gab es zum einen bei der Beschaffung der notwendigen Hardware. So haben wir uns am Anfang viel damit auseinandergesetzt, wie man überhaupt Wasserschäden im Keller messen kann. Es gab die Überlegungen Sensoren zu verwenden, die die relative Feuchtigkeit im Beton messen würden, aber dafür müsste man ein Loch in den Boden bohren und das wäre natürlich nicht so schön. Außerdem fallen diese Art von Sensoren normalerweise recht teuer aus. Später haben wir uns auf klassischen Luftfeuchtigkeitssensor entschieden, der die relative Luftfeuchtigkeit misst. Das reicht eigentlich auch aus, da damit quasi schon der Wassergehalt in der Luft angegeben wird und wenn dieser Wert zu lange in einem hohen Bereich liegt, dann kann es passieren, dass sich der Beton wie ein Schwamm vollsaugt und sich dadurch Schimmel bildet und damit auch Wasserschäden entstehen.\\
Durch die Unterstützung von Prof. Scheffler, haben wir außerdem noch zwei Development-Boards mit LoRa-Modulen erhalten. Dies hat uns eine Menge Arbeit erspart, seperat einen LoRa-Chip und Mikrocontroller bestellen und diese noch konfigurieren zu müssen. \\
Jedoch hatte ich am Anfang, da ich Linux als Betriebssystem verwende, einige Schwierigkeiten mit der Installation des Pymakr-Plugins für den Upload bzw. das Flashen des Codes auf den Mikrocontroller. Somit musste ich ständig den Code via FTP auf das Board rüberkopieren, was sehr umständlich war, da man zuerst einen FTP-Server auf dem Mikrocontroller einrichten, der wiederum eine IP-Addresse von dem AP/Router bekommen musste... Aber Schlussendlich hat es dann doch geklappt und wir konnten, bis auf das LoRaWAN, unsere Projektidee realisieren.


\section{Ausblick} \label{Ausblick}

\subsubsection{Erweiterung von LoRa-LoRa zu LoRaWAN}
In unserem Projekt haben wir bisher leider kein richtiges Sensornetzwerk aufgebaut, sondern zwei Microcontroller über eine P2P LoRa Verbindung Kommunizieren lassen. Dies hat zum einen den Grund, dass wir selber kein LoRa Gateway aufbauen wollten, zum anderen aber auch kein umliegendes Gateway in erreichbarer Nähe haben.

\subsubsection{Benutzung von mehreren Sensorknoten an einem LoRa-Gateway}
Wenn wir es hinbekommen sollten ein eigenes Gateway bereitzustellen, welches auch unsere Publisher-Software implementiert hat, würden wir gerne mehrere Sensorknoten mit diesem Gateway verbinden, um so möglicherweise Unterschiede in den Messdaten zwischen den Microcontrollern zu entdecken.

\subsubsection{Erweiterung der Software zur Einbindung von mehreren Sensorkomponenten}
Derzeit haben bieten wir nur die Möglichkeit einen oder mehrere DHT Sensoren einzubinden. Wir würden gerne eine universelle Schnittstelle bereitstellen, die es ermöglicht eine Vielzahl von Sensoren anzubinden und auszuwerten.

\subsubsection{Batteriebetrieb in Hardware umsetzen}
Den Batteriebetrieb haben wir bisher nur Simuliert. Da wir keine geeigneten Konstruktion bauen konnten, die einen Batteriehalter ordnungsgemäß halten könnte. Zu einem Serienreifen Produkt gehört somit auch ein Gehäuse mit Battriehalter.

\subsubsection{Alternative Dashboard Möglichkeiten}
Wir würden gerne weitere Dashboardmöglichkeiten evalueiren, da die kostenlose Version der \textit{adafruit.io} Dashboards doch recht limiert ist. Ggf. ist es auch Sinnvoll eine eigene Dashboard Plattfrom bereitzustellen, da wir weiterhin den Ansatz der freien Inhalte verfolgen wollen.

\section{Abschlusswort} \label{Abschlusswort}

Peace

