%%
%%  Hochschule für Technik und Wirtschaft Berlin --  Abschlussarbeit
%%
%%  Hauptdokument
%%
%%%%%%%%%%%%%%%%%%%%%%%%%%%%%%%%%%%%%%%%%%%%%%%%%%%

%% Einstellungen und Anpassungen
\input{settings/settings}  		% Import der Pakete und Stil des Dokuments
\input{settings/adjustments} 	% Weitere Pakete und Anpassungen (Sprache, Quellenverwaltung, etc.)
\input{settings/titlepage}		% Layout der Titelseite

%%%%%%%%%%%%%%%%%%%%%%%%%%%%%%%%%%%%%%%%%%%%%
%%%%%%%%%%%%%%%%%%%%%%%%%%%%%%%%%%%%%%%%%%%%%
%% In diesem Bereich müssen Sie Anpassungen für das Deckblatt der Arbeit vornehmen!
%
%% Titel und Author 
\titel{Luftfeuchtigkeits-Sensornetzwerk zur zeitnahen Detektion von Wasserschäden auf Basis von LoRa(WAN)}
\autor{Sidney Göhler und Ilja Buschujew}
% \matrikelnr{s0559016}
%% Version und Abgabedatum
\version{0.2$\alpha$} 	%ToDo: wird derzeit noch nicht genutzt
\datum{25.02.2022}   	% Abgabedatum der Arbeit
%% Typ der Arbeit
\thesistyp{Projektabschlussbericht}
%\thesistyp{Bachelorarbeit}
\abschluss{Projekt Netzbasierte Systeme}
%\abschluss{Bachelor of Engineering (B. Eng.)}
%% Betreuer
\firstExaminer{Prof.~Dr. Thomas Scheffler}
\secondExaminer{Hanna Full}
%% Fachbereich
\fachbereich{1 -- Energie und Information --}
\studiengang{Informations- und Kommunikationstechnik (M. Eng.)}
%%%%%%%%%%%%%%%%%%%%%%%%%%%%%%%%%%%%%%%%%%%%%
%%%%%%%%%%%%%%%%%%%%%%%%%%%%%%%%%%%%%%%%%%%%%
%%%%%%%%%%%%%%%%%%%%%%%%%%%%%%%%%%%%%%%%%%%%%s

%% Pfad zu den Bildern
\graphicspath{
  {pictures/},
}

%%%%%%%%%%%%%%%%%%%%%%%%%%%%%%%%%%%%%%%%%%%%%

%% Start des Dokuments
\begin{document}

%% Deckblatt erzeugen
\maketitle

%% Inhaltsverzeichnis erstellen
\cleardoubleoddpage
\pagenumbering{Roman}
\tableofcontents \clearpage

%%%%%%%%%%%%%%%%%%%%%%%%%%%%%%%%%%%%%%%%%%%%%
%%%%%%%%%%%%%%%%%%%%%%%%%%%%%%%%%%%%%%%%%%%%%
%% In diesem Bereich müssen Sie Anpassungen für den Inhalt der Arbeit vornehmen!
%% Kurzzusammenfassung
\input{abstract/Abstract_de.tex}
\input{abstract/Abstract_en.tex}
\clearpage

%% Hauptteil
\cleardoubleoddpage
\pagenumbering{arabic}


% !TEX root = ../Thesis.tex
%%
%%  Hochschule für Technik und Wirtschaft Berlin --  Abschlussarbeit
%%
%% Kapitel 1
%%
%%

\chapter{Einleitung} \label{Einleitung}

\section{Vorstellung der Projektidee} \label{Vorstellung der Projektidee}

Die Digitalisierung hat unsere Art und Weise wie die Gesellschaft lebt und wie verrichtete Arbeit wertgeschätzt wird, grundlegend verändert. Es sind nicht mehr die Menschen, sondern Computer und Maschinen, die den Takt vorgeben und die Maßstäbe setzen. Arbeit und soziales Zusammenleben werden in einer kapitalistischen Gesellschaft durch die Digitalisierung neu bestimmt. Begriffe wie Homeoffice und Telearbeit sind aus unserem heutigen Arbeitsleben kaum mehr wegzudenken, was schlussendlich in unserer globalisierten Welt zu einem Optimierungswahn geführt hat. Weitere Folgen sind unter anderem die Privatisierung von Wissen und Information, sowie die Ausbeutung von Mensch und Natur.

Aus diesen und weiteren Gründen wünschen sich immer mehr Menschen einen Rückschritt zu einer Gesellschafft, bei der moralische Werte über den wirtschaftlichen Erfolg gestellt werden.
Sie wünschen sich mehr Selbstbestimmung, unter Rücksichtnahme der vorhandenen natürlichen Ressourcen und beteiligten Personen, um Schlussendlich die vorherrschende Ellenbogengesellschaft durch eine sozialere auszutauschen.

Im Diskurs werden unter Anderem Begrenzung Anderer, Grenzsetzung gegenüber Anderen, aber auch Ausgrenzung Anderer bzw. die eigene Ausgrenzung thematisiert und in Frage gestellt, wodurch sich unter anderem die Bewegung der Urban Commoner herauskristallisiert hat.
Urban Commons zielt auf eine Entwicklung von individuellen und gesellschaftlichen Werte und Normen, auf Basis eines Zusammenschlusses einzelner Individuen, um ein bestimmte Gebiete oder eine bestimmte Ressource unabhängig zu gestalten.

Wir, Ilja Buschujew und Sidney Göhler, möchten mit unserem Luftfeuchtigkeit-Temperatur-Sensor Netzwerk unseren Beitrag dazu leisten, um prinzipiell jedem die Möglichkeit zu bieten, seine eigenen Daten zu sammeln und diese mit seinem Umfeld zu teilen, um dem sich fortschreitenden Konkurenzgedanken innerhalb seines Umfeldes entgegen zu wirken ohne dabei auf seine individuellen Bedürfnisse verzichten zu müssen.

\section{Herangehensweise}

In Kapitel \ref{Grundlagen} werden zuerst zentrale Begriffe, wie Smart Objects und Internet"=of"=Things, erläutert. Verschiedenen Technologien und Standards, die im Zusammenhang mit der Aufgabenstellung verwendet werden, werden vorgestellt. Diese Technologien werden eingeteilt in Übertragungstechniken und Kommunikationstechnologien. Eine Auswahl an Applikationen wird erläutert, die für den Zugriff auf die zu implementierende Lösung verwendet werden können. Betriebssysteme und Softwareumgebungen werden vorgestellt, die für die Bearbeitung der Aufgabenstellung in Frage kommen. Im Abschluss werden allgemein technische und nicht"=technische Herausforderungen behandelt.

Im Kapitel \ref{Konzept} wird anhand einer Nutzwertanalyse die Komplettlösung ermittelt, die für die Aufgabenstellung am geeignetsten ist.  Dazu werden zuerst, unabhängig von bestimmten Technologien, Anforderungen beschrieben, die an eine implementierte Lösung gestellt werden. Die Erfüllung dieser Anforderungen wird am Ende dieser Arbeit geprüft.  Am Ende wird die Entscheidung hergeleitet, welche Lösung beispielhaft implementiert wird.

Die Beschreibung der eigentlichen Implementierung findet sich in Kapitel \ref{Implementierung}.

Nach Abschluss der Implementierung wurden Funktionstests der implementierten Applikationen durchgeführt. Der Testaufbau und die Funktionstests werden in Kapitel \ref{Tests} behandelt.

Zum Ende der Arbeit wird die implementierte Lösung kritisch begutachtet. In Kapitel \ref{Fazit} wird geprüft, welche vorher beschriebenen Anforderungen erfüllt wurden. Der Nutzwert der implementierten Lösung wird ermittelt und mit dem Ergebnis der vorherigen Nutzwertanalyse verglichen. Verschiedene Anwendungsmöglichkeiten für die Implementierung und ähnliche Lösungsansätze werden behandelt. Ein Resümee der Arbeit wird gezogen.






\input{chapter/02_Grundlagen.tex}
\input{chapter/03_Konzept.tex}
\input{chapter/04_Implementierung.tex}
\input{chapter/05_Tests.tex}
\input{chapter/06_Fazit.tex}

%% Anhang
\cleardoubleoddpage
\appendix
% !TEX root = ../Thesis.tex
%%
%%  Hochschule für Technik und Wirtschaft Berlin --  Projektabschlussbericht
%%
%% Anhang
%%
%%%%%%%%%%%%%%%%%%%%%%%%%%%%%%%%%%%%%%%%%%%%%%%%%%%%%%%%%%%%%%%%%%%%%


\chapter{Anhang}


\ldots

%%%%%%%%%%%%%%%%%%%%%%%%%%%%%%%%%%%%%%%%%%%%%
%%%%%%%%%%%%%%%%%%%%%%%%%%%%%%%%%%%%%%%%%%%%%
%%%%%%%%%%%%%%%%%%%%%%%%%%%%%%%%%%%%%%%%%%%%%


%% Abkürzungsverzeichnis
% 	Befehl: makeindex Thesis.nlo -s nomencl.ist -o Thesis.nls
%
\input{references/acronyms.tex}
\cleardoublepage
\markboth{\nomname}{\nomname}
\printnomenclature

%% Abbildungsverzeichnis
\listoffigures \clearpage

%% Tabellenverzeichnis
\listoftables \clearpage

%% Quelltextverzeichnis
\lstlistoflistings \clearpage

%% Stichwortverzeichnis
%	Befehl: makeindex Thesis
%
\printindex \clearpage

%% Literaturverzeichnis
%	Befehl: biber Thesis
%
\printbibliography[heading=bibintoc, title={\babel{Literaturverzeichnis}{Bibliography}}]
\clearpage
%Literaturverzeichnisse getrennt nach Stichwort
%\printbibliography[heading=bibintoc, keyword={book}, title={Literaturverzeichnis}]\clearpage
%\printbibliography[heading=bibintoc, keyword={online}, title={Onlinequellen}]\clearpage
%\printbibliography[heading=bibintoc, keyword={image}, title={Bildquellen}]\clearpage

%% Erklärung zur Eigenständigkeit
% !TEX root = Thesis.tex
%%
%%  Hochschule für Technik und Wirtschaft Berlin --  Abschlussarbeit
%%
%%  Erklärung zur Eigenständigkeit
%%
%%%%%%%%%%%%%%%%%%%%%%%%%%%%%%%%%%%%%%%%%%%%%%%%%%%%

\addchap{Eigenständigkeitserklärung}

Hiermit versichern wir, dass wir das vorliegende \thethesistyp{} selbstständig und nur unter
Verwendung der angegebenen Quellen und Hilfsmittel verfasst haben. 

\vskip 1cm

Berlin, den \thedatum

\vskip 1.5cm

\theautor


\end{document}
