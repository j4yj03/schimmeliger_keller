\documentclass[12pt,utf8]{beamer}
\usetheme{Warsaw}
\usecolortheme{default}
\setbeamercovered{transparent}

%Wichtige Standard Pakete!
\usepackage[ngerman]{babel}
\usepackage{xcolor}
\usepackage{graphicx}

%------------------------------------------------------------
%This block of code defines the information to appear in the
%Title page
\title[LSN zur zeitnahen Detektion von Wasserschäden]{Luftfeuchtigkeits-Sensor-Netzwerk zur zeitnahen Detektion von Wasserschäden auf Basis von LoRa(WAN)}
%\subtitle{subtitle}
\author{Sidney Göhler \\ Ilja Buschujew}
\institute[HTW Berlin]{Projekt Netzbasierte Systeme\\
Informations- und Kommunikationstechnik (M. Eng.)\\
Hochschule für Technik und Wirtschaft Berlin}
\date[ProNeSy WS 21/22] % (optional)
{11. Februar 2022}

%End of title page configuration block
%------------------------------------------------------------

% %------------------------------------------------------------
% %The next block of commands puts the table of contents at the 
% %beginning of each section and highlights the current section:
% 
% \AtBeginSection[]
% {
%   \begin{frame}
%     \frametitle{Gliederung}
%     \tableofcontents[currentsection]
%   \end{frame}
% }
% 
% %------------------------------------------------------------

\begin{document}
%The next statement creates the title page.
\frame{\titlepage}
%---------------------------------------------------------
%This block of code is for the table of contents after
%the title page
\begin{frame}
\frametitle{Gliederung}
\tableofcontents
\end{frame}

%---------------------------------------------------------

\section{Einleitung}
\begin{frame}
\frametitle{Einleitung}

\end{frame}

\section{Vorstellung der Projektidee}
\begin{frame}
\frametitle{Vorstellung der Projektidee}
asd
\end{frame}

\subsection{Prinzip des Urban-Commons}
\begin{frame}
\frametitle{Prinzip des Urban-Commons}
asd
\end{frame}

\section{Hardware}
\begin{frame}
\frametitle{Hardware}

\end{frame}
\subsection{Blockdiagramm LoPy4}
\begin{frame}
\frametitle{Blockdiagramm LoPy4}

\end{frame}

\subsection{Schaltplan}
\begin{frame}
\frametitle{Schaltplan}

\end{frame}

\section{Software}
\begin{frame}
\frametitle{Software}

\end{frame}

\subsection{Blockschaltbild}
\begin{frame}
\frametitle{Blockschaltbild}
asd
\end{frame}


\section{Vorführung}
\begin{frame}
\frametitle{Vorführung}

\end{frame}

\section{Ausblick}
\begin{frame}
\frametitle{Ausblick}

\end{frame}


\end{document}
